\documentclass{assignment}
\ProjectInfos{光电子技术}{PHYS6651P}{2021-2022学年第一学期}{第七章作业}{}{陈稼霖}[https://github.com/Chen-Jialin]{SA21038052}

\begin{document}
\begin{prob}
    将 $50$ mW 的光注入 $300$ m 长的光纤中. 如果在另一端受到的功率为 $30$ mW,试问每公里光纤的损耗是多少(用 dB/km 表示)?如果光纤长 $5$ 公里,输出功率将是多少?
\end{prob}
\begin{sol}
    光纤的损耗为
    \begin{align}
        \alpha=\frac{1}{L}10\lg\frac{I}{I_0}=\frac{1}{0.3\text{ km}}10\lg\frac{30\text{ mW}}{50\text{ mW}}=-7.39\text{ dB/km}.
    \end{align}
    若光纤长 $5$ km,输出功率为
    \begin{align}
        I=I_0*10^{-5\text{ km}\times\alpha/10}=1.00\times 10^{-2}\text{ mW}=10.0\,\mu\text{W}.
    \end{align}
\end{sol}

\begin{prob}
    用截去法测量光纤在 $0.85\,\mu$m 波长的损耗. 若用光电接收器测得 $2$ 公里长的光纤输出电压为 $2.1$ V,当光纤被截去剩下 $3$ 米时输出电压增加到 $10.5$ V,求每公里该光纤在 $0.85\,\mu$m 波长上的衰减,并用公式不确定度 $=\pm 0.2/(L_1-L_2)$ dB/km 来估计测量精度.
\end{prob}
\begin{sol}
    
\end{sol}

\begin{prob}
    分析影响单模光纤色散的各种因素,如何减小单模光纤中的色散?
\end{prob}
\begin{ans}
    
\end{ans}
\end{document}