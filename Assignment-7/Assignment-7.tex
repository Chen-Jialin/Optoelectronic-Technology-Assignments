\documentclass{assignment}
\ProjectInfos{光电子技术}{PHYS6651P}{2021-2022学年第一学期}{第十一章作业}{}{陈稼霖}[https://github.com/Chen-Jialin]{SA21038052}

\begin{document}
\begin{prob}
    \begin{itemize}
        \item[(a)] 若没有镜面或其他光反馈的部件的光发射器,能够由受激发射产生光吗?这些光相干吗?
        \item[(b)] 解释激光器中阈值现象的意义;
        \item[(c)] 为什么场限制激光器具有较低的阈值电流和较高的效率?
    \end{itemize}
\end{prob}
\begin{ans}
    \begin{itemize}
        \item[(1)] 能, 这些光相干, 例如放大器就没有镜面或其他光反馈的部件, 其中的介质被泵浦光激励时, 可在信号光的作用下受激发射, 从而放大信号光.
        \item[(2)] 阈值现象: 增益增大到一定程度时, 激光器的工作模式从非受激辐射转变为受激辐射.
        \item[(3)] 阈值电流密度
        \begin{align}
            J_{\text{th}}=\frac{8\pi n^2\Delta\nu D}{\eta_{\text{in}}\lambda^2}\left(\alpha+\frac{1}{2L}\ln\frac{1}{2R}\right),
        \end{align}
        其中 $n$ - 介质折射率, $\Delta\nu$ - 纵模频率间隔, $D$ - 光发射层厚度, $\eta_{\text{in}}$ - 内量子效率, $\lambda$ - 激光波长, $\alpha$ - 谐振腔内损耗系数, $L$ - 谐振腔长, $R$ - 谐振腔端面功率反射系数. 场限制激光器可将发光层厚度 $D$ 限制在一个较小的范围内, 从而可得较低的阈值电流 $J_{\text{th}}$, 以及较高的效率.
    \end{itemize}
\end{ans}

\begin{prob}
    工作波长 $\lambda_0=8950$ \AA 的 \ce{GaAs} DFB 激光器,若用一级光栅,试求光栅间距为多少?
\end{prob}
\begin{sol}
    光栅间距
    \begin{align}
        \Lambda=\frac{\lambda}{2n}=\frac{895.0\text{ nm}}{2\times 3.6}=124.3\text{ nm}.
    \end{align}
\end{sol}

\begin{prob}
    如果 \ce{GaAs} 介质的折射率 $n=3.6$,试求 \ce{GaAs} 半导体激光器谐振腔端面的反射率 $R$.
\end{prob}
\begin{sol}
    由菲涅尔公式, 谐振腔端面的反射率
    \begin{align}
        R=\left(\frac{n-1}{n+1}\right)^2=\left(\frac{3.6-1}{3.6+1}\right)^2=0.32.
    \end{align}
\end{sol}

\begin{prob}
    半导体激光器的发散角可近似为 $\theta\approx\lambda_0/a$,$a$ 为有源区线度,若 $d=2\,\mu$m,$w=12\,\mu$m,求该激光器的发散角 $\theta_{\perp}$ 和 $\theta_{\parallel}$ 的值.
\end{prob}
\begin{sol}
    该激光的发散角
    \begin{align}
        \theta_{\perp}=&\frac{\lambda_0}{d},\\
        \theta_{\parallel}=&\frac{\lambda_0}{w}.
    \end{align}
\end{sol}
\end{document}